%-*- coding: utf-8 -*-
\documentclass[11pt]{article}

%Paquets
\usepackage[frenchb]{babel}
\usepackage[T1]{fontenc}
\usepackage[utf8]{inputenc}
\usepackage{amsmath}
\usepackage{amssymb}
\usepackage{enumerate}
\usepackage{amsthm}
\usepackage{mathrsfs}
\usepackage{answers}
\usepackage{lipsum}
\usepackage[margei=probastat,margep=probastat,miroirp]{feuilletd}
\frenchbsetup{StandardEnumerateEnv=true}


%%%%%%
% \usepackage{showframe}
%%%%%%


%Police
\usepackage{fouriernc}


\usepackage{fancyhdr}
\pagestyle{fancy}
\fancyfoot{}
\lhead{\bfseries \Large Exercise sheet}
\rhead{\bfseries Exercises}
\renewcommand{\footrulewidth}{0.4pt}
\rfoot{UTBM}
\setlength{\headheight}{18.6pt}



\usepackage{tikz}
\newcounter{numExo}
\newenvironment{exercice}[1][]{\refstepcounter{numExo}
\noindent
\begin{trivlist}\hypertarget{exercice \thenumExo}
\item\raisebox{-0.3em}{\begin{tikzpicture}[x=1.0cm,y=1.0cm]
    \draw[color=blue!60,fill=blue!50!black, rounded corners=3pt] (0,0) 
                rectangle (0.7, 0.5); 
    \node at (0.35,0.25) {\textcolor{white}{\bfseries\arabic{numExo}}};
    \node[overlay] at (.35,.25) {\hyperlink{correction 
\thenumExo}{\phantom{\rule{.7cm}{.5cm}}}};
\end{tikzpicture}}~\textcolor{blue}{\sl #1}\vspace{0.1em}}
{\end{trivlist}}



\Newassociation{correction}{CorrectionExoTemp}{correction}
\renewenvironment{CorrectionExoTemp}[1]{%
\noindent
\begin{trivlist}\hypertarget{correction #1}
\item\raisebox{-0.3em}{\begin{tikzpicture}[x=1.0cm,y=1.0cm]
    \draw[color=orange!60,fill=orange, rounded corners=3pt] (0,0) 
                rectangle (0.7, 0.5); 
    \node at (0.35,0.25) {\textcolor{white}{\bfseries#1}};
    \node[overlay] at (.35,.25) {\hyperlink{exercice 
#1}{\phantom{\rule{.7cm}{.5cm}}}};
\end{tikzpicture}}~\vspace{0.1em}
}
{\end{trivlist}}

\usepackage{hyperref}
\hypersetup{ 
    hidelinks
}

\begin{document}

\Opensolutionfile{correction}


\begin{exercice}
 \lipsum[1]
 \begin{correction}
  \lipsum[2]
 \end{correction}
\end{exercice}


\begin{exercice}
 \lipsum[1]
 \begin{correction}
  \lipsum[2]
 \end{correction}
\end{exercice}


\begin{exercice}
 \lipsum[1]
 \begin{correction}
  \lipsum[2]
 \end{correction}
\end{exercice}


\begin{exercice}
 \lipsum[1]
 \begin{correction}
  \lipsum[2]
 \end{correction}
\end{exercice}


\begin{exercice}
 \lipsum[1]
 \begin{correction}
  \lipsum[2]
 \end{correction}
\end{exercice}



\Closesolutionfile{correction}
\newpage 
\rhead{\bfseries Solutions}
\begin{CorrectionExoTemp}{1}
  \lipsum[2]
 
\end{CorrectionExoTemp}
\begin{CorrectionExoTemp}{2}
  \lipsum[2]
 
\end{CorrectionExoTemp}
\begin{CorrectionExoTemp}{3}
  \lipsum[2]
 
\end{CorrectionExoTemp}
\begin{CorrectionExoTemp}{4}
  \lipsum[2]
 
\end{CorrectionExoTemp}
\begin{CorrectionExoTemp}{5}
  \lipsum[2]
 
\end{CorrectionExoTemp}


\end{document}

